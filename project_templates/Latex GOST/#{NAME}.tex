% 1. Стиль и язык
\documentclass[utf8x]{G7-32} % Стиль (по умолчанию будет 14pt)
\usepackage[T2A]{fontenc}
\usepackage[english, russian]{babel}
\usepackage{amsmath}
\usepackage{xcolor}
\usepackage{accsupp}
\usepackage{listings}
\usepackage[unicode, colorlinks=true, linkcolor=blue]{hyperref}
\usepackage{geometry}
\usepackage{graphicx}
\usepackage{latexsym}
\usepackage{amssymb}
\usepackage{indentfirst}
\usepackage{tikz}
\usepackage{pgfplots}

\pgfplotsset{compat=1.7}

\definecolor{codegreen}{rgb}{0,0.6,0}
\definecolor{codegray}{rgb}{0.5,0.5,0.5}
\definecolor{codepurple}{rgb}{0.58,0,0.82}
\definecolor{backcolour}{rgb}{0.95,0.95,0.92}
\newcommand{\noncopynumber}[1]{%
    \BeginAccSupp{method=escape,ActualText={}}%
    #1%
    \EndAccSupp{}%
}
\lstdefinestyle{mycodestyle}{
    frame=single,
    backgroundcolor=\color{backcolour},
    commentstyle=\color{codegreen},
    keywordstyle=\color{blue},
    numbers=left,
    numberstyle=\color{codegray},
    breaklines=true,
    basicstyle=\ttfamily,
    breakatwhitespace=false,
    captionpos=b,
    keepspaces=true,
    showspaces=false,
    showstringspaces=false,
    showtabs=false,
    columns=fullflexible,
    numberstyle=\noncopynumber,
    language=C++,
    alsolanguage=PHP
}
\lstset{style=mycodestyle}

\graphicspath{{img/}}

\begin{document}

% Остальные стандартные настройки убраны в preamble.inc.tex.
\include{preamble.inc}

% Настройки листингов.
\include{listings.inc}

% Полезные макросы листингов.
\include{macros.inc}

\frontmatter % выключает нумерацию ВСЕГО; здесь начинаются ненумерованные главы: реферат, введение, глоссарий, сокращения и прочее.

% Команды \breakingbeforechapters и \nonbreakingbeforechapters
% управляют разрывом страницы перед главами.
% По-умолчанию страница разрывается.

% \nobreakingbeforechapters
% \breakingbeforechapters

\begin{center}
    {\scshape ITMO University\par}
    \vspace{80mm}
    {\large\bfseries\scshape #{NAME}\par}
    \vspace{20mm}
    {\small\scshape Кербер Егор\par}
    \vfill
    {\footnotesize\scshape\today\par}
    \thispagestyle{empty}
\end{center}

\include{00-abstract}

\tableofcontents

\include{10-defines}
\include{11-abbrev}

\include{12-intro}

% Тут идет предисловие
% Тут идет предисловие
% Тут идет предисловие
% Тут идет предисловие
% Тут идет предисловие
% Тут идет предисловие
% Тут идет предисловие

\mainmatter % это включает нумерацию глав и секций в документе ниже

\include{20-analysis}
\include{30-design}
\include{40-impl}
\include{50-research}

% Тут идет основная часть
% Тут идет основная часть
% Тут идет основная часть
% Тут идет основная часть
% Тут идет основная часть
% Тут идет основная часть
% Тут идет основная часть

\backmatter %% Здесь заканчивается нумерованная часть документа и начинаются ссылки и
            %% заключение

\include{80-conclusion}

% Тут вывод
% Тут вывод
% Тут вывод
% Тут вывод
% Тут вывод
% Тут вывод
% Тут вывод

\include{81-biblio}

\appendix   % Тут идут приложения

\include{90-appendix1}
\include{91-appendix2}

\end{document}

